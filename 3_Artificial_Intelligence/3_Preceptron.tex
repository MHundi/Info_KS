% Options for packages loaded elsewhere
\PassOptionsToPackage{unicode}{hyperref}
\PassOptionsToPackage{hyphens}{url}
\PassOptionsToPackage{dvipsnames,svgnames,x11names}{xcolor}
%
\documentclass[
  11pt,
  a4paper,
  DIV=11,
  numbers=noendperiod]{scrartcl}

\usepackage{amsmath,amssymb}
\usepackage{iftex}
\ifPDFTeX
  \usepackage[T1]{fontenc}
  \usepackage[utf8]{inputenc}
  \usepackage{textcomp} % provide euro and other symbols
\else % if luatex or xetex
  \usepackage{unicode-math}
  \defaultfontfeatures{Scale=MatchLowercase}
  \defaultfontfeatures[\rmfamily]{Ligatures=TeX,Scale=1}
\fi
\usepackage{lmodern}
\ifPDFTeX\else  
    % xetex/luatex font selection
    \setmainfont[]{Avenir Next}
\fi
% Use upquote if available, for straight quotes in verbatim environments
\IfFileExists{upquote.sty}{\usepackage{upquote}}{}
\IfFileExists{microtype.sty}{% use microtype if available
  \usepackage[]{microtype}
  \UseMicrotypeSet[protrusion]{basicmath} % disable protrusion for tt fonts
}{}
\makeatletter
\@ifundefined{KOMAClassName}{% if non-KOMA class
  \IfFileExists{parskip.sty}{%
    \usepackage{parskip}
  }{% else
    \setlength{\parindent}{0pt}
    \setlength{\parskip}{6pt plus 2pt minus 1pt}}
}{% if KOMA class
  \KOMAoptions{parskip=half}}
\makeatother
\usepackage{xcolor}
\setlength{\emergencystretch}{3em} % prevent overfull lines
\setcounter{secnumdepth}{-\maxdimen} % remove section numbering
% Make \paragraph and \subparagraph free-standing
\makeatletter
\ifx\paragraph\undefined\else
  \let\oldparagraph\paragraph
  \renewcommand{\paragraph}{
    \@ifstar
      \xxxParagraphStar
      \xxxParagraphNoStar
  }
  \newcommand{\xxxParagraphStar}[1]{\oldparagraph*{#1}\mbox{}}
  \newcommand{\xxxParagraphNoStar}[1]{\oldparagraph{#1}\mbox{}}
\fi
\ifx\subparagraph\undefined\else
  \let\oldsubparagraph\subparagraph
  \renewcommand{\subparagraph}{
    \@ifstar
      \xxxSubParagraphStar
      \xxxSubParagraphNoStar
  }
  \newcommand{\xxxSubParagraphStar}[1]{\oldsubparagraph*{#1}\mbox{}}
  \newcommand{\xxxSubParagraphNoStar}[1]{\oldsubparagraph{#1}\mbox{}}
\fi
\makeatother


\providecommand{\tightlist}{%
  \setlength{\itemsep}{0pt}\setlength{\parskip}{0pt}}\usepackage{longtable,booktabs,array}
\usepackage{calc} % for calculating minipage widths
% Correct order of tables after \paragraph or \subparagraph
\usepackage{etoolbox}
\makeatletter
\patchcmd\longtable{\par}{\if@noskipsec\mbox{}\fi\par}{}{}
\makeatother
% Allow footnotes in longtable head/foot
\IfFileExists{footnotehyper.sty}{\usepackage{footnotehyper}}{\usepackage{footnote}}
\makesavenoteenv{longtable}
\usepackage{graphicx}
\makeatletter
\newsavebox\pandoc@box
\newcommand*\pandocbounded[1]{% scales image to fit in text height/width
  \sbox\pandoc@box{#1}%
  \Gscale@div\@tempa{\textheight}{\dimexpr\ht\pandoc@box+\dp\pandoc@box\relax}%
  \Gscale@div\@tempb{\linewidth}{\wd\pandoc@box}%
  \ifdim\@tempb\p@<\@tempa\p@\let\@tempa\@tempb\fi% select the smaller of both
  \ifdim\@tempa\p@<\p@\scalebox{\@tempa}{\usebox\pandoc@box}%
  \else\usebox{\pandoc@box}%
  \fi%
}
% Set default figure placement to htbp
\def\fps@figure{htbp}
\makeatother

\usepackage[document]{ragged2e}
\KOMAoption{captions}{tableheading}
\makeatletter
\@ifpackageloaded{caption}{}{\usepackage{caption}}
\AtBeginDocument{%
\ifdefined\contentsname
  \renewcommand*\contentsname{Table of contents}
\else
  \newcommand\contentsname{Table of contents}
\fi
\ifdefined\listfigurename
  \renewcommand*\listfigurename{List of Figures}
\else
  \newcommand\listfigurename{List of Figures}
\fi
\ifdefined\listtablename
  \renewcommand*\listtablename{List of Tables}
\else
  \newcommand\listtablename{List of Tables}
\fi
\ifdefined\figurename
  \renewcommand*\figurename{Figure}
\else
  \newcommand\figurename{Figure}
\fi
\ifdefined\tablename
  \renewcommand*\tablename{Table}
\else
  \newcommand\tablename{Table}
\fi
}
\@ifpackageloaded{float}{}{\usepackage{float}}
\floatstyle{ruled}
\@ifundefined{c@chapter}{\newfloat{codelisting}{h}{lop}}{\newfloat{codelisting}{h}{lop}[chapter]}
\floatname{codelisting}{Listing}
\newcommand*\listoflistings{\listof{codelisting}{List of Listings}}
\makeatother
\makeatletter
\makeatother
\makeatletter
\@ifpackageloaded{caption}{}{\usepackage{caption}}
\@ifpackageloaded{subcaption}{}{\usepackage{subcaption}}
\makeatother

\usepackage{bookmark}

\IfFileExists{xurl.sty}{\usepackage{xurl}}{} % add URL line breaks if available
\urlstyle{same} % disable monospaced font for URLs
\hypersetup{
  pdftitle={Perceptron - Influencer},
  colorlinks=true,
  linkcolor={blue},
  filecolor={Maroon},
  citecolor={Blue},
  urlcolor={Blue},
  pdfcreator={LaTeX via pandoc}}


\title{Perceptron - Influencer}
\author{}
\date{}

\begin{document}
\maketitle


\subsubsection{Import der benötigten
Bibliotheken}\label{import-der-benuxf6tigten-bibliotheken}

\subsubsection{Definition der Gewichte und des
Lernfaktors}\label{definition-der-gewichte-und-des-lernfaktors}

\subsubsection{Hilfsfunktionen}\label{hilfsfunktionen}

\paragraph{a) Resetfunktion der
Gewichte}\label{a-resetfunktion-der-gewichte}

\paragraph{b) Rohsumme aus den
Vektoren}\label{b-rohsumme-aus-den-vektoren}

\paragraph{c) Lernalgorithmus}\label{c-lernalgorithmus}

\paragraph{d) Wertvorhersage Funktion}\label{d-wertvorhersage-funktion}

\subsubsection{Lernzyklus}\label{lernzyklus}

\paragraph{Erster Lernzyklus}\label{erster-lernzyklus}

\begin{verbatim}
Die Gewichte (Bias, Alter, IQ) sind [-0.5, -20.0, -65.0]
\end{verbatim}

\begin{verbatim}
[-0.5, -20.0, -65.0]
\end{verbatim}

\paragraph{Zwanzig weitere Lernzyklen}\label{zwanzig-weitere-lernzyklen}

\begin{verbatim}
Die Gewichte (Bias, Alter, IQ) sind [0.0, -10.0, -10.0]
Die Gewichte (Bias, Alter, IQ) sind [0.0, -20.0, -20.0]
Die Gewichte (Bias, Alter, IQ) sind [0.0, -30.0, -30.0]
Die Gewichte (Bias, Alter, IQ) sind [0.0, -40.0, -40.0]
Die Gewichte (Bias, Alter, IQ) sind [0.0, -50.0, -50.0]
Die Gewichte (Bias, Alter, IQ) sind [0.5, -40.0, 5.0]
Die Gewichte (Bias, Alter, IQ) sind [0.5, -50.0, -5.0]
Die Gewichte (Bias, Alter, IQ) sind [0.5, -60.0, -15.0]
Die Gewichte (Bias, Alter, IQ) sind [0.5, -70.0, -25.0]
Die Gewichte (Bias, Alter, IQ) sind [0.5, -80.0, -35.0]
Die Gewichte (Bias, Alter, IQ) sind [1.0, -70.0, 20.0]
Die Gewichte (Bias, Alter, IQ) sind [1.0, -70.0, 20.0]
Die Gewichte (Bias, Alter, IQ) sind [1.0, -70.0, 20.0]
Die Gewichte (Bias, Alter, IQ) sind [1.0, -70.0, 20.0]
Die Gewichte (Bias, Alter, IQ) sind [1.0, -70.0, 20.0]
Die Gewichte (Bias, Alter, IQ) sind [1.0, -70.0, 20.0]
Die Gewichte (Bias, Alter, IQ) sind [1.0, -70.0, 20.0]
Die Gewichte (Bias, Alter, IQ) sind [1.0, -70.0, 20.0]
Die Gewichte (Bias, Alter, IQ) sind [1.0, -70.0, 20.0]
Die Gewichte (Bias, Alter, IQ) sind [1.0, -70.0, 20.0]
\end{verbatim}

\subsubsection{Entscheidungsgerade}\label{entscheidungsgerade}

Die Gewichte ändern sich nicht mehr. Wir erhalten die Ungleichuchung,
die entscheidet, ob ein Datenpunkt zu den Influencer Followern gehört
oder nicht.

\[ +1 -7\cdot \text{Alter} +20\cdot \text{IQ} \geq 0\]

Mit x = Alter und y = IQ folgt: \[ +1 -7\cdot x +20\cdot y \geq 0\]

Umformung der Ungleichung nach y ergibt: \[ y\geq 3,5x-\frac{1}{20}\]

Als Grenzgerade erhält man: \[ y = 3,5x-\frac{1}{20}\]

\paragraph{Veranschaulichung}\label{veranschaulichung}

\pandocbounded{\includegraphics[keepaspectratio]{3_Preceptron_files/figure-pdf/cell-11-output-1.pdf}}




\end{document}
